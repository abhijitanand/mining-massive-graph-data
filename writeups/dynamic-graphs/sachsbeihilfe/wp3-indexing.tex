\workpackage{Indexing and Query Processing on large evolving graphs}
\label{wp:indexing}

\textbf{Matching-based Queries} 
We consider two types of queries : \emph{standing matching queries} and \emph{adhoc matching queries}. Standing queries (also called as publisher-subscriber queries) are a static set of queries which are checked for satisfaction in the face of changing or dynamic input data. Examples are dense-subgraph queries proposed by Angel et. al.~\cite{angel_dense_2013} which reports dense subgraphs whenever streaming edges result in a subgraph becoming substantially dense. The query in that case is a desired density coefficient. In our scenario, we could also be interested in potentially new matchings given a stream of incoming edges/nodes. The second class of queries are the more standard adhoc queries, where user can desire a given type of matching on a given subgraph. As an example, for time-varying graphs a user could be interested in matchings for the graph which was valid in a given time interval.


\subsection{Research Questions}     
In this respect we can formulate the following research questions which we intend to answer:
\begin{itemize}
    \item \textsf{[RQ I]} How can we \emph{efficiently construct} indexes specialized for answering matching queries ? 

    \item \textsf{[RQ II]} What kind of index-maintenance strategies is needed to be employed to avoid partial or complete index recomputations ?

    \item \textsf{[RQ II]} What query processing techniques need to be employed for both \emph{standing} and \emph{adhoc} matching queries ?

\end{itemize}



