% !TEX TS-program = pdflatex
% !TEX encoding = UTF-8 Unicode

\documentclass[11pt]{article} 

\usepackage[show]{notes-alt}
\usepackage{authordate1-4}
\usepackage[utf8]{inputenc}
\usepackage{booktabs}
\usepackage{xspace}
\usepackage{float}
\usepackage{hyperref}
\usepackage{eurosym}
\usepackage{multirow}
\usepackage{graphicx}


\usepackage[scaled=1.0]{helvet}
\renewcommand{\familydefault}{\sfdefault}

%%% PAGE DIMENSIONS
\usepackage{geometry} 
\geometry{a4paper}
 \geometry{margin=2.94cm} 

\setcounter{tocdepth}{4}
\setcounter{secnumdepth}{4}
\usepackage[english]{babel}

\usepackage{graphicx}

%%% PACKAGES
\usepackage{booktabs} 
\usepackage{array} 
\usepackage{paralist} 
\usepackage{verbatim}
\usepackage{subfig} 

%%% HEADERS & FOOTERS
\usepackage{fancyhdr} 
\pagestyle{fancy}
\renewcommand{\headrulewidth}{0pt} 
\lhead{}\chead{}\rhead{}
\lfoot{}\cfoot{\thepage}\rfoot{}

%own workpackage counter
\newcounter{wcounter}

%and definition
\newcommand{\workpackage}[1]{%
	\refstepcounter{wcounter}
    \par\vspace{.95cm}\noindent\textsc{Workpackage \arabic{wcounter}: #1}% 
    \par\vspace{.3cm}
}


%%% SECTION TITLE APPEARANCE
\usepackage{sectsty}
\allsectionsfont{\sffamily\mdseries\upshape} % (See the fntguide.pdf for font help)
% (This matches ConTeXt defaults)

%%% ToC (table of contents) APPEARANCE
\usepackage[nottoc,notlof,notlot]{tocbibind} % Put the bibliography in the ToC
\usepackage[titles,subfigure]{tocloft} % Alter the style of the Table of Contents
\renewcommand{\cftsecfont}{\rmfamily\mdseries\upshape}
\renewcommand{\cftsecpagefont}{\rmfamily\mdseries\upshape} % No bold!


%some own markup for comments and todos
%\newcommand{\todo}[1]{{\textcolor{red}{TODO: #1}}}
\renewcommand{\comment}[1]{{\textcolor{blue}{Comment: #1}}}

\usepackage{tikz}
\usepackage{gantt}
\usepackage{multicol}
\date{}

\title{Semi-streaming models for Sparse evolving graphs}

% \title{Pantheon: Efficiently Creating and Maintaining \\Semantically Meaningful Entity Rankings \\ at Large Scale}
\author{Megha Khosla \\
Avishek Anand
}


\begin{document}

\maketitle

\begin{tabbing}
xxx\= \kill
{\bf 
Name and Address
}
\end{tabbing}

\begin{centering}

Graphs encode the interconnections between entities to model relations between them.They have proven to be an intuitive datastructure in modelling, querying and reasoning about real-life relationships. They have been heavily used in link analysis for Web search, social networks analysis, community detection, biological networks and in knowlege representation. There is however two important characteristics of such real world graphs. First, graphs are progressively becoming \emph{massive} and secondly, most of them evolve with time making them \emph{dynamic}. ~\cite{Henzinger,Feigenbaum, Alon}

\end{centering}

\inote{try}

\note{trying notes}

\todo{trying todo}


%\subsubsection {Temporal Graphs }
%
%\subsection{Project Related Publications}
%
%\section{Objectives and Work Programme}
%\subsection{Anticipated Total Duration of the Project}
%\subsection{Objectives}
%\subsection{Work Programme including Proposed Research Methods}
%\subsection{Data Handling}
%\subsection{Other Information}
%\subsection{Explanations on the Proposed Investigations}
%\subsection{Information on the Scientific and Financial Involvement of International Cooperation Partners}


%
\section{State of the art and preliminary work}


\subsection*{Preliminary Work}


\subsection{Project-related publications}


\subsubsection{Articles published by outlets with scientific quality assurance, book publications, and works accepted for publication but not yet published.}
		

		
		\begin{itemize}
		\item 
		\end{itemize}

\subsubsection{Other publications}


\begin{itemize}
\item 
\end{itemize}


\subsubsection{Patents}

	\paragraph{Pending}

	\paragraph{Issued}


\input{current}

\newpage

\section{Objectives and work programme}

	\subsection{Anticipated total duration of the project}
		36 months
\subsection{Objectives}


\subsection{Work programme incl. proposed research methods}

The work in this project is split into three work packages. WP\ref{wp:generation}   addresses the
generation of entity rankings, while WP\ref{wp:maintenance}  considers ranking maintenance and event reporting.
The research achievements obtained in these two work packages will be used in WP\ref{wp:implementation}  to implement a full-fledged
prototype system that is going to be used further to create a repository of entity rankings over a variety of information sources.

 
\begin{itemize}
\item WP1: Composition of Semantically Meaningful Rankings
\item WP2: Ranking Staleness, Maintenance, and Reporting
\item WP3: Prototype Implementation, Repository Creation, and Benchmarking 
\end{itemize}


\workpackage{Composition of Semantically Meaningful Rankings}
\label{wp:generation}



\workpackage{Algorithms in the Semi-streaming model}
\label{wp:maintenance}

In~\cite{ballsbins} we proposed an algorithm, which we call \emph{local search allocation}, finds maximum cardinality matching in linear time with high probability. Our approach in~\cite{ballsbins} operates by assigning labels to vertices in $V$. The vertices for augmenting path are chosen based on their labels. The labels are so updated such that at any time the label of a vertex is atmost the shortest distance to an unmatched vertex. In addition to the efficiency of the approach, the algorithm never runs in loops and gives the information when the vertex cannot be matched. The time complexity of the algorithm is $O(|E|)$ and the space complexity is $O(|V|)$. This approach is different from the greedy approach which is commonly employed in such a model in the sense that it allows the matching to change as and when necessary. This means that once an edge is selected for the matching, it can be kicked out later on as and when required.

Even though this algorithm has been designed for a more specific class of bipartite graphs, it can be applied directly on more general bipartite graphs. We focus on the following research questions.

With (i) $O(n poly \log n)$ space and  (ii) amortized $O(1)$ time processing of each edge
\begin{itemize}
\item  \textsf{[RQ I]} Can we achieve an approximation ration closer to the known lower bound $(e/(e-1))$ for general bipartite graphs ?
\item \textsf{[RQ II]} What approximation ratios can we achieve with high probability on different models of random graphs like uniformly random or power law graphs ?
\item \textsf{[RQ III]} What approximation ratios can we achieve for weighted $b$ matchings in bipartite graphs?

\end{itemize}



\workpackage{Prototype Implementation and Utilization}
\label{wp:implementation}

\textbf{Matching-based Queries} 
We consider two types of queries : \emph{standing matching queries} and \emph{adhoc matching queries}. Standing queries (also called as publisher-subscriber queries) are a static set of queries which are checked for satisfaction in the face of changing or dynamic input data. Examples are dense-subgraph queries proposed by Angel et. al.~\cite{angel_dense_2013} which reports dense subgraphs whenever streaming edges result in a subgraph becoming substantially dense. The query in that case is a desired density coefficient. In our scenario, we could also be interested in potentially new matchings given a stream of incoming edges/nodes. The second class of queries are the more standard adhoc queries, where user can desire a given type of matching on a given subgraph. As an example, for time-varying graphs a user could be interested in matchings for the graph which was valid in a given time interval.


\subsection{Research Questions}     
In this respect we can formulate the following research questions which we intend to answer:
\begin{itemize}
    \item \textsf{[RQ I]} How can we \emph{efficiently construct} indexes specialized for answering matching queries ? 

    \item \textsf{[RQ II]} What kind of index-maintenance strategies is needed to be employed to avoid partial or complete index recomputations ?

    \item \textsf{[RQ II]} What query processing techniques need to be employed for both \emph{standing} and \emph{adhoc} matching queries ?

\end{itemize}



\subsection*{Work Organization}


\begin{figure}
\centering
\scalebox{0.7}{
  \begin{gantt}{11}{12}
    \begin{ganttitle}
     \titleelement{Year 1}{4}
     \titleelement{Year 2}{4}
     \titleelement{Year 3}{4}
    \end{ganttitle}
    \begin{ganttitle}
     \titleelement{Q1}{1}
     \titleelement{Q2}{1}
     \titleelement{Q3}{1}
     \titleelement{Q4}{1}
     \titleelement{Q1}{1}
     \titleelement{Q2}{1}
     \titleelement{Q3}{1}
     \titleelement{Q4}{1}
     \titleelement{Q1}{1}
     \titleelement{Q2}{1}
     \titleelement{Q3}{1}
     \titleelement{Q4}{1}
    \end{ganttitle}
    \ganttbar[color=green!50]{{\bf WP1:} Understanding Rankings}{0}{6}
    \ganttbar[color=green!50]{{\bf WP1:} Role of Attributes}{3}{6}
    \ganttbar[color=green!50]{{\bf WP1:} Ranking Generation}{4}{8}

    \ganttbar[color=red!50]{{\bf WP2:} Staleness}{0}{4}
    \ganttbar[color=red!50]{{\bf WP2:} Maintenance}{3}{5}
    \ganttbar[color=red!50]{{\bf WP2:} Event Ranking}{5}{6}
    
    \ganttbar[color=blue!50]{{\bf WP3:} Implementation}{2}{8}
    \ganttbar[color=blue!50]{{\bf WP3:} Repository Creation and UI}{4}{8}
    \ganttbar[color=blue!50]{{\bf WP3:} Quality \& Performance Benchmarks}{6}{5}

%    \ganttmilestone[color=blue]{Milestone 1}{8}

  \end{gantt}
}
\caption{Rough  chronological alignment of the individual tasks }
\label{fig:gantt}
\end{figure}


	\subsection{Data handling}


	\subsection{Other information}

None.

	\subsection{Descriptions of proposed investigations involving experiments on humans, human materials or animals}
		Does not apply.
	\subsection{Information on scientific and financial involvement of international cooperation partners}
		Does not apply.



%%%%%%%%%%%%%%%
%%%%%%%%%%%%%%%%%%

\section{Bibliography}

%%%%%%%%%%%%%%%%%%

%\begin{multicols}{2}
%\small
% \bibliographystyle{authordate1}
	
% %\bibliographystyle{plain}
% \bibliography{../proposal-outline}

	%\bibliography{main}
%\end{multicols}

%%%%%%%%%%%%%%%%%%

 


%\bibliographystyle{authordate1}
\bibliographystyle{plain}
\bibliography{references}

\input{funding-sm}

\end{document}
