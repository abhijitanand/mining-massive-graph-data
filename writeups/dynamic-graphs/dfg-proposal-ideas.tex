\documentclass{scrartcl}

\usepackage{tikz}
\usepackage{booktabs}
\usepackage{paralist} \usepackage {subfigure} \usepackage {graphicx}
\usepackage{amsfonts, algorithm, algorithmic, graphicx, subfigure, tabularx, amssymb, amsmath, rotating, multirow}
\usepackage[show]{notes-alt}

%% Custom Commands
% \usepackage{times}
\usepackage{xspace}
\newcommand{\aanand}[1]{\textcolor{blue}{\textit{#1}}}
\newcommand{\comment}{}
\newcommand{\hide}[1]{}
\usepackage{xspace}
\newcommand {\N}{\mathcal{N}}
\newcommand{\argmax}{\operatornamewithlimits{argmax}}

\pagenumbering{arabic}

\begin{document}

  
 \title{Querying and Mining Graph Data Streams}
 \maketitle

\section{Introduction}

Graphs encode the interconnections between entities to model relations between them.They have proven to be an intuitive datastructure in modelling, querying and reasoning about real-life relationships. In websearch the graphs derived from links between webpages are used for link analysis and establishing authority of webpages. In social networks, the graph structure captures interesting insights about the influential nodes, community structures, connectivity and reachability etc. In biological networks people are interested in how diseases spread. In ontologies, they are useful to model taxonomies in the form of type hierarchies. Apart from these scenarios they find applications in road, water networks for shortest path computations and flow problems.

Traditional research has focussed on largely on \emph{static graphs}. However, as one might quickly realize, most of the scenarios enumerated above have a evolutionary nature to them. Social networks, Web graphs, citation networks and even biological networks evolve over time. So there has been a growing need to support data management, querying, mining and analytics tasks over evolving or time-varying graphs.  Moreso, to date there is little work on understanding the challenges needed to ingest, manage and query such data in large graph databases. In this proposal we attempt to scope out the problems relating to dynamic graphs and more specifically time-varying graphs with an intent to identify open issues and problems which would be worth investigating.

\todo{Certain examples neededlike shortest paths, k core decomposition, matchings}

\section{State of the Art and Preliminary Work}

We divide the related work survey into two sections: (a) algorithmic foundations on streaming graph primitives, and (b) indexing and querying for streaming graphs.

\subsection{State of the Art}

\note{add something}


\subsection{Preliminary Work}

\todo{
what we did ? How our work would be used in this project
for example :  
\begin{itemize}
	\item (Megha ) Studied  k cores , dense components, matching
	\item(Avishek) :  
\end{itemize}

}


\subsubsection {k cores }
A bipartite graph can been seen as a hypergraph. 
In \cite{kCores} we estimate the threshold for the density of the random uniform hypergraph when the k core first appears. Such a threshold is associated with the probability of existence of perfect matchings in teh associated bipartite graphs. Such estimates are quite useful .... 

\subsubsection {Matching in a Streaming Model}

%matching is core to resource allocation problems of various types from the scheduling and Operations Research literature, for example, allocating jobs to machines in cloud computing
Two natural variants of this problem have been considered in the literature: (1) the edge
arrival setting, where edges arrive in the stream and (2) the vertex arrival setting, where vertices on one
side of the graph arrive in the stream together with all their incident edges.
The latter setting has also
been studied extensively in the context of online algorithms, where each arriving vertex has to either be
matched irrevocably or discarded upon arrival.

The basic algorithms for finding matchings rely on the definition of augmenting paths:
given a matching $M$ in the graph, an augmenting path is an oddlength
(simple) path with its edges alternating between being in $M$
and not, and with the two end edges not in $M$. In order to increase the size of the matching, one starts with an unmatched vertex and walks along the augmenting path to reach another unmatched vertex. The unmached edges of the previous matching now are the matched edges and vice versa.

One can employ various methods to find these augmenting paths; namely breadth first search (BFS), depth first search (DFS) and random walk. The BFS and DFS approaches require the full knowledge of the graph and hence are not ideal for streaming models. The random walk method on the one hand chooses each vertex on the augmenting path randomly and is a stateless approach but on the other hand has no gurantees of the convergence in the worst case. 

The other class of algorithms employed for finding online matchings is \emph{greedy} approach where each edge is included in the matching if it does not destroy the matching. At present this is the best single pass algorithm providing $2$ approximation when the edges are streamed in an arbitrary order. The strongest known lower bound for the single pass algorithms is $e/(e-1)\approx 1.58$ which also applies when the edges are grouped by end point~\cite{boundsKap,Goel}.

In the model that we consider, we are given a bipartite graph $G(U,V,E)$, in which $U$ is known to the algorithm, vertices in $V$ are
unknown, but arrive one at a time, revealing the edges incident on
them as they arrive. The algorithm has to match a vertex
as soon as it arrives. 
In addition each vertex in $V$ has $k$ incident edges and the other end of such edges are chosen uniformly randomly and independently. This model is motivated from a class of multiple choice load balancing problems in which each job ( $u\in U$) chooses $k$ machines uniformly randomly and independently. In \cite{kCores} we prove that there exists a threshold such that when the density of $G$ is below that threshold a (left-) perfect matching (i.e. all jobs are assigned to one of their choices) exists, otherwise not. In~\cite{ballsbins} we propose an algorithm which we call \emph{local search allocation} finds maximum cardinality matching in linear time with high probability.
Our approach in~\cite{ballsbins} operates by assigning labels to vertices in $V$. The vertices for augmenting path ae chosen based on their labels. The labels are so updated such that at any time the label of a vertex is atmost the shortest distance to an unmatched vertex. In addition to the efficiency of the approach, the algorithm never runs in loops and gives the information when the vertex cannot be matched. Emperical evidence suggests that our approach performs an order of magnitude better than the random walk method.



As a starting point our proposed research focusses on the following questions.

\begin{enumerate}
\item Can we extend/adapt local search allocation algorithm to find approximate maximum cardinality matchings in the streaming model  in which a bipartite graph is given by a stream of edges in an arbitrary order? Such a semi streaming algoritm has memory $O(n~ \text{polylog} ~n)$, and the graph vertices are known before processing the stream of edges.
\item 
\end{enumerate}


state bounds for maximum cardinality matchings in bipartite graphs and general graphs
explain greedy algorithm ... variants of greedy algorithm improving bounds..not much improvement theretically
Introduce local search allocation for online setting....the model is different from the above considered models...a new idea based on approximate augmenting path lengths ; simple to implement 
scope to adapt the algorithm to general graphs ; there are of course problems in breaking the symmetry
we aim to develop new algorithms based on local search method proposed in .... for computing aproximate matchings for online model

We consider a streaming model 

\subsection{Indexing and Query Processing}

Within the streaming model, there are important applications of matching algorithms like \emph{internet advertisements}, \emph{load balancing in cloud computing}, \emph{switch scheduling} etc. Based on the research problems in the previous section, we now turn our attention to how to utilize those results into building indexing frameworks to efficiently answer \emph{matching-based queries} or simply \emph{mathcing queries}. 


In graph database research, the input data is typically organized into an index optimized for a certain kind of query. As an example, a specialized index can be created over an input graph to answer \emph{reachability queries}, i.e., if a pair of nodes are reachable from each other. Similarly, we intend to design efficient indexing methods for streaming graph input to find matchings.

\subsubsection{Matching-based Queries} 
We consider two types of queries : \emph{standing matching queries} and \emph{adhoc matching queries}. Standing queries (also called as publisher-subscriber queries) are a static set of queries which are checked for satisfaction in the face of changing or dynamic input data. Examples are dense-subgraph queries proposed by Angel et. al.~\cite{} which reports dense subgraphs whenever streaming edges result in a subgraph becoming substantially dense. The query in that case is a desired density coefficient. In our scenario, we could also be interested in potentially new matchings given a stream of incoming edges/nodes. The second class of queries are the more stndard adhoc queries, where user can desire a given type of matching on a given subgraph. As an example, for time-varying graphs a user could be interested in matchings for the graph which was valid in a given time interval.


\subsubsection{Research Questions} 
In this respect we can formulate the following research questions which we intend to answer:
\begin{itemize}
	\item \textsf{[RQ I]} How can we \emph{efficiently construct} indexes specialized for answering matching queries ? 

	\item \textsf{[RQ II]} What kind of index-maintenance strategies is needed to be employed to avoid partial or complete index recomp utations ?

	\item \textsf{[RQ II]} What query processing techniques need to be employed for both \emph{standing} and \emph{adhoc} matching queries ?

\end{itemize}

\subsubsection {Temporal Graphs }

\subsection{Project Related Publications}

\section{Objectives and Work Programme}
\subsection{Anticipated Total Duration of the Project}
\subsection{Objectives}
\subsection{Work Programme including Proposed Research Methods}
\subsection{Data Handling}
\subsection{Other Information}
\subsection{Explanations on the Proposed Investigations}
\subsection{Information on the Scientific and Financial Involvement of International Cooperation Partners}

\bibliographystyle{plain}
\bibliography{dfg-proposal-ideas}

%General intro to graphs and contemporary application domains
 

% Evolving graphs and need for scoping out the area



%\section{Temporal Graphs}
\label{sec:temporal-graphs}

%General intro to graphs and contemporary application domains
Graphs encode the interconnections between entities to model relations between them. They have proven to be an intuitive datastructure in modelling, querying and reasoning about real-life relationships. In websearch the graphs derived from links between webpages are used for link analysis and establishing authority of webpages. In social networks, the graph structure captures interesting insights about the influential nodes, community structures, connectivity and reachability etc. In biological networks people are interested in how diseases spread. In ontologies, they are useful to model taxonomies in the form of type hierarchies. Apart from these scenarios they find applications in road, water networks for shortest path computations and flow problems. 

% Evolving graphs and need for scoping out the area
Traditional research has focussed on largely on \emph{static graphs}. However, as one might quickly realize, most of the scenarios enumerated above have a evolutionary nature to them. Social networks, Web graphs, citation networks and even biological networks evolve over time. So there has been a growing need to support data management, querying, mining and analytics tasks over evolving or time-varying graphs. In this proposal we attempt to scope out the problems relating to dynamic graphs and more specifically time-varying graphs with an intent to identify open issues and problems which would be worth investigating.

% Our focus areas
We study dynamic graphs with a specific focus on three dimensions -- \emph{data or input models} for representing graphs, support for \emph{primitive operations} in querying and mining dynamic graphs, and finally data management issues pertaining to storing dynamic graphs for supporting \emph{approximate} and \emph{exact queries}.

\subsection{Dynamic Graph Input/Data Model}
\label{sec: data-model}

In this section we outline some of the prevalent data models in the literature pertaining to dynamic graphs. 

\begin{enumerate}
	\item {\textbf{Streaming Model: }} The most prevalent and standard data model for dynamic graphs is the \emph{streaming model} which assumes the input to be a sequence edges from a given family of nodes. (cite Muthukrishnan et. al Data Streams: Algorithms and Applications). Here the algorithm sees the entire input and the typical assumption is of a limited working memory. 

%The relevant questions for such a model are : how d
	
	
	\item {\textbf{Dynamic Graph Model :}} In this setting, a graph changes over time and the goal is to keep track of the changes so as to be able to efficiently answer graph queries. The main difference is that here when a change is performed to a graph, the algorithm is notified of the change. (cite Demetrescu et. al from "Dynamic graph algorithms").


	\item {\textbf{Property Testing Model :}}  Here the goal is to find whether a graph has some property or is far from satisfying the property using a limited number of queries. (cite the Ron guy from "Algorithmic and analysis techniques in property testing.")

	\item {\textbf{Mutli-armed Bandit Model :}}  In the standard multi- armed bandit setting there are k slot machines (one-armed bandits) and pulling a lever in a slot machine gives a re- ward, which depends on the machine, and reveals informa- tion about the machine. The objective is to select the machines to query so as to maximize the total reward; as in our case, the number of queries in every time step is limited. (cite Muthukrishnan et. al Data Streams: Algorithms and Applications)

	\item {\textbf{Fixed-probe Model :}} The time is assumed to proceed in discrete steps, numbered by positive integers. At each time step t, the data is given by a (possibly weighted) graph Gt. The data is changing gradually, i.e., the graph Gt+1 is obtained from Gt by a small random change.2 At each time step t, the algorithm is allowed to probe a small portion of the graph Gt, and then must output a solution for the problem under consideration. We would like this solu- tion to be close to the“correct”solution for the graph Gt. In this paper we do not impose any constraint on the amount of memory the algorithm maintains or the running time of the algorithm, although all of the algorithms we present are quite efficient with respect to these factors. (Cite Aris from "Algorithms for Evolving Graphs").

	\item {\textbf{Parametric optimization and kinetic problems :}} In the parametric optimization model, the edge weights are known continuous functions of a real parameter $\lambda$ (often referred to as “time”), and the goal is to identify how the solution changes as $\lambda$ varies. A kinetic problem combines parametric optimizations and dynamic data structures for insertions and deletions. In such a problem, at the begin- ning a parametric problem of parameter $\lambda$ is given; as the time $\lambda$ progresses, the weight functions change and objects (e.g., edges) are inserted or deleted. The goal is to efficiently maintain the optimal solution at each point in time.
(cite Henzinger from Parametric and kinetic minimum spanning trees.)


\end{enumerate}



\subsection{Querying and Mining Dynamic Graphs}
\label{sec:query-mining}

In this section we investigate the primitive operations that have been supported on dynamic graphs over the past few years. In doing so, we identify potential open problems which need to be addressed. 


\subsubsection{Path Related Problems} 
\label{sec:path-problems}


\subsubsection{Sub-graph Problems}
\label{sec:subgraph-problems}


\subsubsection{Matching Problems}
\label{sec:matching-problems}


\subsection{Data Structures for Approximate and Exact Queries}
\label{sec:approx-exacts}

\subsubsection{Index Structures based on Graph Sketches}
\label{sec:indexing-dynamic-graphs}

\subsubsection{Spanners, Sparsifiers and Expanders}
\label{sec:spanners}

\subsubsection{Approximate operations on Graphs}
\label{sec:approximate-problems}




%\section{Open Problems}


%\section{Seminal Works}


%\section{Application Areas}

\end{document}

